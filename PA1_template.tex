\documentclass[]{article}
\usepackage{lmodern}
\usepackage{amssymb,amsmath}
\usepackage{ifxetex,ifluatex}
\usepackage{fixltx2e} % provides \textsubscript
\ifnum 0\ifxetex 1\fi\ifluatex 1\fi=0 % if pdftex
  \usepackage[T1]{fontenc}
  \usepackage[utf8]{inputenc}
\else % if luatex or xelatex
  \ifxetex
    \usepackage{mathspec}
  \else
    \usepackage{fontspec}
  \fi
  \defaultfontfeatures{Ligatures=TeX,Scale=MatchLowercase}
\fi
% use upquote if available, for straight quotes in verbatim environments
\IfFileExists{upquote.sty}{\usepackage{upquote}}{}
% use microtype if available
\IfFileExists{microtype.sty}{%
\usepackage{microtype}
\UseMicrotypeSet[protrusion]{basicmath} % disable protrusion for tt fonts
}{}
\usepackage[margin=1in]{geometry}
\usepackage{hyperref}
\hypersetup{unicode=true,
            pdftitle={PA1\_template.Rmd},
            pdfauthor={Venkat},
            pdfborder={0 0 0},
            breaklinks=true}
\urlstyle{same}  % don't use monospace font for urls
\usepackage{color}
\usepackage{fancyvrb}
\newcommand{\VerbBar}{|}
\newcommand{\VERB}{\Verb[commandchars=\\\{\}]}
\DefineVerbatimEnvironment{Highlighting}{Verbatim}{commandchars=\\\{\}}
% Add ',fontsize=\small' for more characters per line
\usepackage{framed}
\definecolor{shadecolor}{RGB}{248,248,248}
\newenvironment{Shaded}{\begin{snugshade}}{\end{snugshade}}
\newcommand{\AlertTok}[1]{\textcolor[rgb]{0.94,0.16,0.16}{#1}}
\newcommand{\AnnotationTok}[1]{\textcolor[rgb]{0.56,0.35,0.01}{\textbf{\textit{#1}}}}
\newcommand{\AttributeTok}[1]{\textcolor[rgb]{0.77,0.63,0.00}{#1}}
\newcommand{\BaseNTok}[1]{\textcolor[rgb]{0.00,0.00,0.81}{#1}}
\newcommand{\BuiltInTok}[1]{#1}
\newcommand{\CharTok}[1]{\textcolor[rgb]{0.31,0.60,0.02}{#1}}
\newcommand{\CommentTok}[1]{\textcolor[rgb]{0.56,0.35,0.01}{\textit{#1}}}
\newcommand{\CommentVarTok}[1]{\textcolor[rgb]{0.56,0.35,0.01}{\textbf{\textit{#1}}}}
\newcommand{\ConstantTok}[1]{\textcolor[rgb]{0.00,0.00,0.00}{#1}}
\newcommand{\ControlFlowTok}[1]{\textcolor[rgb]{0.13,0.29,0.53}{\textbf{#1}}}
\newcommand{\DataTypeTok}[1]{\textcolor[rgb]{0.13,0.29,0.53}{#1}}
\newcommand{\DecValTok}[1]{\textcolor[rgb]{0.00,0.00,0.81}{#1}}
\newcommand{\DocumentationTok}[1]{\textcolor[rgb]{0.56,0.35,0.01}{\textbf{\textit{#1}}}}
\newcommand{\ErrorTok}[1]{\textcolor[rgb]{0.64,0.00,0.00}{\textbf{#1}}}
\newcommand{\ExtensionTok}[1]{#1}
\newcommand{\FloatTok}[1]{\textcolor[rgb]{0.00,0.00,0.81}{#1}}
\newcommand{\FunctionTok}[1]{\textcolor[rgb]{0.00,0.00,0.00}{#1}}
\newcommand{\ImportTok}[1]{#1}
\newcommand{\InformationTok}[1]{\textcolor[rgb]{0.56,0.35,0.01}{\textbf{\textit{#1}}}}
\newcommand{\KeywordTok}[1]{\textcolor[rgb]{0.13,0.29,0.53}{\textbf{#1}}}
\newcommand{\NormalTok}[1]{#1}
\newcommand{\OperatorTok}[1]{\textcolor[rgb]{0.81,0.36,0.00}{\textbf{#1}}}
\newcommand{\OtherTok}[1]{\textcolor[rgb]{0.56,0.35,0.01}{#1}}
\newcommand{\PreprocessorTok}[1]{\textcolor[rgb]{0.56,0.35,0.01}{\textit{#1}}}
\newcommand{\RegionMarkerTok}[1]{#1}
\newcommand{\SpecialCharTok}[1]{\textcolor[rgb]{0.00,0.00,0.00}{#1}}
\newcommand{\SpecialStringTok}[1]{\textcolor[rgb]{0.31,0.60,0.02}{#1}}
\newcommand{\StringTok}[1]{\textcolor[rgb]{0.31,0.60,0.02}{#1}}
\newcommand{\VariableTok}[1]{\textcolor[rgb]{0.00,0.00,0.00}{#1}}
\newcommand{\VerbatimStringTok}[1]{\textcolor[rgb]{0.31,0.60,0.02}{#1}}
\newcommand{\WarningTok}[1]{\textcolor[rgb]{0.56,0.35,0.01}{\textbf{\textit{#1}}}}
\usepackage{graphicx,grffile}
\makeatletter
\def\maxwidth{\ifdim\Gin@nat@width>\linewidth\linewidth\else\Gin@nat@width\fi}
\def\maxheight{\ifdim\Gin@nat@height>\textheight\textheight\else\Gin@nat@height\fi}
\makeatother
% Scale images if necessary, so that they will not overflow the page
% margins by default, and it is still possible to overwrite the defaults
% using explicit options in \includegraphics[width, height, ...]{}
\setkeys{Gin}{width=\maxwidth,height=\maxheight,keepaspectratio}
\IfFileExists{parskip.sty}{%
\usepackage{parskip}
}{% else
\setlength{\parindent}{0pt}
\setlength{\parskip}{6pt plus 2pt minus 1pt}
}
\setlength{\emergencystretch}{3em}  % prevent overfull lines
\providecommand{\tightlist}{%
  \setlength{\itemsep}{0pt}\setlength{\parskip}{0pt}}
\setcounter{secnumdepth}{0}
% Redefines (sub)paragraphs to behave more like sections
\ifx\paragraph\undefined\else
\let\oldparagraph\paragraph
\renewcommand{\paragraph}[1]{\oldparagraph{#1}\mbox{}}
\fi
\ifx\subparagraph\undefined\else
\let\oldsubparagraph\subparagraph
\renewcommand{\subparagraph}[1]{\oldsubparagraph{#1}\mbox{}}
\fi

%%% Use protect on footnotes to avoid problems with footnotes in titles
\let\rmarkdownfootnote\footnote%
\def\footnote{\protect\rmarkdownfootnote}

%%% Change title format to be more compact
\usepackage{titling}

% Create subtitle command for use in maketitle
\providecommand{\subtitle}[1]{
  \posttitle{
    \begin{center}\large#1\end{center}
    }
}

\setlength{\droptitle}{-2em}

  \title{PA1\_template.Rmd}
    \pretitle{\vspace{\droptitle}\centering\huge}
  \posttitle{\par}
    \author{Venkat}
    \preauthor{\centering\large\emph}
  \postauthor{\par}
      \predate{\centering\large\emph}
  \postdate{\par}
    \date{7/26/2019}


\begin{document}
\maketitle

\hypertarget{reproducibleresearch-week2-assignment-for-peer-review}{%
\section{ReproducibleResearch: Week2 Assignment for Peer
Review}\label{reproducibleresearch-week2-assignment-for-peer-review}}

\hypertarget{step1-load-and-read-data.}{%
\subsection{Step1: Load and Read
Data.}\label{step1-load-and-read-data.}}

\begin{Shaded}
\begin{Highlighting}[]
\CommentTok{#download file directly from online and extract it}
\KeywordTok{download.file}\NormalTok{(}\StringTok{"https://d396qusza40orc.cloudfront.net/repdata%2Fdata%2Factivity.zip"}\NormalTok{, }\DataTypeTok{destfile =} \StringTok{"activity.zip"}\NormalTok{, }\DataTypeTok{mode =} \StringTok{"wb"}\NormalTok{)}
\KeywordTok{unzip}\NormalTok{(}\StringTok{"activity.zip"}\NormalTok{)}

\CommentTok{#Read the data and check the contents}
\NormalTok{activity_data <-}\StringTok{ }\KeywordTok{read.csv}\NormalTok{(}\StringTok{"activity.csv"}\NormalTok{, }\DataTypeTok{header =} \OtherTok{TRUE}\NormalTok{)}
\KeywordTok{head}\NormalTok{(activity_data)}
\end{Highlighting}
\end{Shaded}

\begin{verbatim}
##   steps       date interval
## 1    NA 2012-10-01        0
## 2    NA 2012-10-01        5
## 3    NA 2012-10-01       10
## 4    NA 2012-10-01       15
## 5    NA 2012-10-01       20
## 6    NA 2012-10-01       25
\end{verbatim}

\hypertarget{step-2-histogram-of-total-stepsday}{%
\subsection{Step 2: Histogram of total
steps/day}\label{step-2-histogram-of-total-stepsday}}

\begin{Shaded}
\begin{Highlighting}[]
\CommentTok{# aggregate the 5 min data into day level}
\NormalTok{stepsbydate <-}\StringTok{ }\NormalTok{activity_data }\OperatorTok\StringTok{ }\KeywordTok{select}\NormalTok{(date,steps) }\OperatorTok\StringTok{ }\KeywordTok{group_by}\NormalTok{(date) }\OperatorTok\StringTok{ }\KeywordTok{summarize}\NormalTok{(}\DataTypeTok{totalsteps =} \KeywordTok{sum}\NormalTok{(steps)) }\OperatorTok\KeywordTok{na.omit}\NormalTok{()}
\CommentTok{#plot the histogram}
\KeywordTok{hist}\NormalTok{(stepsbydate}\OperatorTok{$}\NormalTok{totalsteps, }\DataTypeTok{xlab=}\StringTok{"Total Daily Steps"}\NormalTok{, }\DataTypeTok{ylab=}\StringTok{"Freq"}\NormalTok{, }\DataTypeTok{main=}\StringTok{"Histogram of Steps by Day"}\NormalTok{, }\DataTypeTok{breaks =} \DecValTok{15}\NormalTok{)}
\end{Highlighting}
\end{Shaded}

\includegraphics{PA1_template_files/figure-latex/unnamed-chunk-3-1.pdf}

\hypertarget{step-3-mean-and-median-of-stepsday}{%
\subsection{Step 3: Mean and Median of
steps/day}\label{step-3-mean-and-median-of-stepsday}}

\begin{Shaded}
\begin{Highlighting}[]
\NormalTok{stepsbydateMean <-}\StringTok{ }\KeywordTok{mean}\NormalTok{(stepsbydate}\OperatorTok{$}\NormalTok{totalsteps)}
\NormalTok{stepsbydateMedian <-}\StringTok{ }\KeywordTok{median}\NormalTok{(stepsbydate}\OperatorTok{$}\NormalTok{totalsteps)}
\end{Highlighting}
\end{Shaded}

\begin{itemize}
\tightlist
\item
  Mean: \ensuremath{1.0766189\times 10^{4}}
\item
  Median: 10765
\end{itemize}

\hypertarget{step-4-time-series-plot-of-avg-steps-taken-per-day}{%
\subsection{Step 4: Time series Plot of Avg Steps taken per
day}\label{step-4-time-series-plot-of-avg-steps-taken-per-day}}

\begin{Shaded}
\begin{Highlighting}[]
\NormalTok{databyinterval <-}\StringTok{ }\NormalTok{activity_data }\OperatorTok\StringTok{ }\KeywordTok{select}\NormalTok{(interval, steps) }\OperatorTok\StringTok{ }\KeywordTok{na.omit}\NormalTok{() }\OperatorTok\StringTok{ }\KeywordTok{group_by}\NormalTok{(interval) }\OperatorTok\StringTok{ }\KeywordTok{summarize}\NormalTok{(}\DataTypeTok{totalsteps =} \KeywordTok{mean}\NormalTok{(steps))}
\CommentTok{## Now plot the summary of steps}
\KeywordTok{ggplot}\NormalTok{(databyinterval, }\KeywordTok{aes}\NormalTok{(}\DataTypeTok{x=}\NormalTok{interval, }\DataTypeTok{y=}\NormalTok{totalsteps))}\OperatorTok{+}\KeywordTok{geom_line}\NormalTok{()}\OperatorTok{+}\KeywordTok{labs}\NormalTok{(}\DataTypeTok{title=}\StringTok{"Time Series Plot"}\NormalTok{, }\DataTypeTok{y=}\StringTok{"Avg # of steps taken"}\NormalTok{, }\DataTypeTok{x=}\StringTok{"interval - 5 min"}\NormalTok{)}
\end{Highlighting}
\end{Shaded}

\includegraphics{PA1_template_files/figure-latex/unnamed-chunk-5-1.pdf}

\hypertarget{step-5-the-5-minute-interval-where-average-contains-max-of-steps}{%
\subsection{Step 5: The 5-minute interval where average contains max
\#of
steps}\label{step-5-the-5-minute-interval-where-average-contains-max-of-steps}}

\begin{Shaded}
\begin{Highlighting}[]
\NormalTok{databyinterval[}\KeywordTok{which}\NormalTok{(databyinterval}\OperatorTok{$}\NormalTok{totalsteps }\OperatorTok{==}\StringTok{ }\KeywordTok{max}\NormalTok{(databyinterval}\OperatorTok{$}\NormalTok{totalsteps)),]}
\end{Highlighting}
\end{Shaded}

\begin{verbatim}
## # A tibble: 1 x 2
##   interval totalsteps
##      <int>      <dbl>
## 1      835       206.
\end{verbatim}

\hypertarget{step-6-code-to-describe-and-show-the-strategy-for-imputing-the-missing-data}{%
\subsection{Step 6: Code to describe and show the strategy for imputing
the missing
data}\label{step-6-code-to-describe-and-show-the-strategy-for-imputing-the-missing-data}}

\begin{Shaded}
\begin{Highlighting}[]
\NormalTok{count_of_missing_values =}\StringTok{ }\KeywordTok{length}\NormalTok{(}\KeywordTok{which}\NormalTok{(}\KeywordTok{is.na}\NormalTok{(activity_data}\OperatorTok{$}\NormalTok{steps)))}
\KeywordTok{summary}\NormalTok{(activity_data)}
\end{Highlighting}
\end{Shaded}

\begin{verbatim}
##      steps                date          interval     
##  Min.   :  0.00   2012-10-01:  288   Min.   :   0.0  
##  1st Qu.:  0.00   2012-10-02:  288   1st Qu.: 588.8  
##  Median :  0.00   2012-10-03:  288   Median :1177.5  
##  Mean   : 37.38   2012-10-04:  288   Mean   :1177.5  
##  3rd Qu.: 12.00   2012-10-05:  288   3rd Qu.:1766.2  
##  Max.   :806.00   2012-10-06:  288   Max.   :2355.0  
##  NA's   :2304     (Other)   :15840
\end{verbatim}

\begin{itemize}
\tightlist
\item
  missing data rows: 2304 Use impute function with mean and fill steps
  for the days they are missing.
\end{itemize}

\begin{Shaded}
\begin{Highlighting}[]
\KeywordTok{library}\NormalTok{(dplyr)}
\NormalTok{replacewithmean <-}\StringTok{ }\ControlFlowTok{function}\NormalTok{(num) }\KeywordTok{replace}\NormalTok{(num, }\KeywordTok{is.na}\NormalTok{(num),}\KeywordTok{mean}\NormalTok{(num, }\DataTypeTok{na.rm =} \OtherTok{TRUE}\NormalTok{))}
\NormalTok{activitydata_nomissing <-}\StringTok{ }\NormalTok{activity_data }\OperatorTok\StringTok{ }\KeywordTok{group_by}\NormalTok{(interval) }\OperatorTok\StringTok{ }\KeywordTok{mutate}\NormalTok{(}\DataTypeTok{steps =} \KeywordTok{replacewithmean}\NormalTok{(steps))}
\KeywordTok{head}\NormalTok{(activitydata_nomissing)}
\end{Highlighting}
\end{Shaded}

\begin{verbatim}
## # A tibble: 6 x 3
## # Groups:   interval [6]
##    steps date       interval
##    <dbl> <fct>         <int>
## 1 1.72   2012-10-01        0
## 2 0.340  2012-10-01        5
## 3 0.132  2012-10-01       10
## 4 0.151  2012-10-01       15
## 5 0.0755 2012-10-01       20
## 6 2.09   2012-10-01       25
\end{verbatim}

\begin{Shaded}
\begin{Highlighting}[]
\NormalTok{new_activity_data =}\StringTok{ }\KeywordTok{as.data.frame}\NormalTok{(activitydata_nomissing)}
\KeywordTok{head}\NormalTok{(new_activity_data)}
\end{Highlighting}
\end{Shaded}

\begin{verbatim}
##       steps       date interval
## 1 1.7169811 2012-10-01        0
## 2 0.3396226 2012-10-01        5
## 3 0.1320755 2012-10-01       10
## 4 0.1509434 2012-10-01       15
## 5 0.0754717 2012-10-01       20
## 6 2.0943396 2012-10-01       25
\end{verbatim}

\begin{Shaded}
\begin{Highlighting}[]
\NormalTok{count_of_missing_values2 =}\StringTok{ }\KeywordTok{length}\NormalTok{(}\KeywordTok{which}\NormalTok{(}\KeywordTok{is.na}\NormalTok{(new_activity_data}\OperatorTok{$}\NormalTok{steps)))}
\end{Highlighting}
\end{Shaded}

\begin{itemize}
\tightlist
\item
  Number of missing values(steps): 0
\end{itemize}

\begin{Shaded}
\begin{Highlighting}[]
\KeywordTok{summary}\NormalTok{(new_activity_data)}
\end{Highlighting}
\end{Shaded}

\begin{verbatim}
##      steps                date          interval     
##  Min.   :  0.00   2012-10-01:  288   Min.   :   0.0  
##  1st Qu.:  0.00   2012-10-02:  288   1st Qu.: 588.8  
##  Median :  0.00   2012-10-03:  288   Median :1177.5  
##  Mean   : 37.38   2012-10-04:  288   Mean   :1177.5  
##  3rd Qu.: 27.00   2012-10-05:  288   3rd Qu.:1766.2  
##  Max.   :806.00   2012-10-06:  288   Max.   :2355.0  
##                   (Other)   :15840
\end{verbatim}

\hypertarget{step-7-histogram-of-total-steps-taken-for-each-day-after-missing-values-are-imputed}{%
\subsection{Step 7: Histogram of Total steps taken for each day after
missing values are
imputed}\label{step-7-histogram-of-total-steps-taken-for-each-day-after-missing-values-are-imputed}}

For the histogram sum up the steps for each day

\begin{Shaded}
\begin{Highlighting}[]
\NormalTok{day_summary <-}\StringTok{ }\KeywordTok{aggregate}\NormalTok{(new_activity_data}\OperatorTok{$}\NormalTok{steps, }\DataTypeTok{by=}\KeywordTok{list}\NormalTok{(new_activity_data}\OperatorTok{$}\NormalTok{date), sum)}

\KeywordTok{names}\NormalTok{(day_summary)[}\DecValTok{1}\NormalTok{] =}\StringTok{ "date"}
\KeywordTok{names}\NormalTok{(day_summary)[}\DecValTok{2}\NormalTok{] =}\StringTok{ "totalsteps"}
\KeywordTok{head}\NormalTok{(day_summary)}
\end{Highlighting}
\end{Shaded}

\begin{verbatim}
##         date totalsteps
## 1 2012-10-01   10766.19
## 2 2012-10-02     126.00
## 3 2012-10-03   11352.00
## 4 2012-10-04   12116.00
## 5 2012-10-05   13294.00
## 6 2012-10-06   15420.00
\end{verbatim}

\begin{Shaded}
\begin{Highlighting}[]
\KeywordTok{hist}\NormalTok{(day_summary}\OperatorTok{$}\NormalTok{totalsteps, }\DataTypeTok{xlab=}\StringTok{"Total Daily Steps"}\NormalTok{,}\DataTypeTok{ylab=}\StringTok{"Freq"}\NormalTok{ ,}\DataTypeTok{main=}\StringTok{"Histogram of Steps by Day after imputatiob"}\NormalTok{, }\DataTypeTok{breaks =} \DecValTok{15}\NormalTok{)}
\end{Highlighting}
\end{Shaded}

\includegraphics{PA1_template_files/figure-latex/unnamed-chunk-13-1.pdf}

\hypertarget{step-8-panel-split-comparing-the-average-number-of-steps-taken-per-5min-interval-across-week-days-and-weekends}{%
\subsection{Step 8: Panel split comparing the average number of steps
taken per 5min interval across week days and
weekends}\label{step-8-panel-split-comparing-the-average-number-of-steps-taken-per-5min-interval-across-week-days-and-weekends}}

\begin{Shaded}
\begin{Highlighting}[]
\NormalTok{new_activity_data}\OperatorTok{$}\NormalTok{weekend_flag <-}\StringTok{ }\KeywordTok{ifelse}\NormalTok{(}\KeywordTok{weekdays}\NormalTok{(}\KeywordTok{as.Date}\NormalTok{(new_activity_data}\OperatorTok{$}\NormalTok{date)) }\OperatorTok\StringTok{ }\KeywordTok{c}\NormalTok{(}\StringTok{"Monday"}\NormalTok{,}\StringTok{"Tuesday"}\NormalTok{,}\StringTok{"Wednesday"}\NormalTok{,}\StringTok{"Thursday"}\NormalTok{,}\StringTok{"Friday"}\NormalTok{),}\StringTok{"Weekday"}\NormalTok{,}\StringTok{"Weekend"}\NormalTok{)}
\KeywordTok{head}\NormalTok{(new_activity_data)}
\end{Highlighting}
\end{Shaded}

\begin{verbatim}
##       steps       date interval weekend_flag
## 1 1.7169811 2012-10-01        0      Weekday
## 2 0.3396226 2012-10-01        5      Weekday
## 3 0.1320755 2012-10-01       10      Weekday
## 4 0.1509434 2012-10-01       15      Weekday
## 5 0.0754717 2012-10-01       20      Weekday
## 6 2.0943396 2012-10-01       25      Weekday
\end{verbatim}

\begin{Shaded}
\begin{Highlighting}[]
\NormalTok{new_activity_data <-}\StringTok{ }\NormalTok{(new_activity_data }\OperatorTok\StringTok{ }\KeywordTok{group_by}\NormalTok{(interval,weekend_flag) }\OperatorTok\StringTok{ }\KeywordTok{summarise}\NormalTok{(}\DataTypeTok{Mean=} \KeywordTok{mean}\NormalTok{(steps)))}

\KeywordTok{ggplot}\NormalTok{(new_activity_data, }\DataTypeTok{mapping =} \KeywordTok{aes}\NormalTok{(}\DataTypeTok{x=}\NormalTok{interval, }\DataTypeTok{y=}\NormalTok{Mean)) }\OperatorTok{+}\StringTok{ }\KeywordTok{geom_line}\NormalTok{()}\OperatorTok{+}
\KeywordTok{facet_grid}\NormalTok{(weekend_flag }\OperatorTok{~}\NormalTok{.) }\OperatorTok{+}\KeywordTok{xlab}\NormalTok{(}\StringTok{"Interval"}\NormalTok{) }\OperatorTok{+}\KeywordTok{ylab}\NormalTok{(}\StringTok{"Mean of steps"}\NormalTok{) }\OperatorTok{+}\StringTok{ }\KeywordTok{ggtitle}\NormalTok{(}\StringTok{"comparision of steps for each interval"}\NormalTok{)}
\end{Highlighting}
\end{Shaded}

\includegraphics{PA1_template_files/figure-latex/unnamed-chunk-15-1.pdf}

\hypertarget{step-9-all-of-r-code-needed-to-reproduce-the-results-in-the-report.}{%
\subsection{Step 9: All of R code needed to reproduce the results in the
report.}\label{step-9-all-of-r-code-needed-to-reproduce-the-results-in-the-report.}}


\end{document}
